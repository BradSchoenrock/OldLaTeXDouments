\documentclass{article}
\usepackage{amsmath,amssymb,slashed,feynmf,sparticles,latexsym}
\usepackage[margin=1.0in]{geometry}
\title{Determining the Cross Section for $\mu^- \mu^+ \rightarrow Higgs \rightarrow e^- e^+$}
\author{Andrew Chegwidden \& Brad Schoenrock}
\date{}
\begin{document}
\maketitle


%%%%Section 1
\section{The Problem}
Calculate the total cross section for the process $\mu^- \mu^+ \rightarrow Higgs \rightarrow e^- e^+$.  This process is important for considering the feasibility of Higgs production at a muon collider.
\newline
\newline
\begin{center}
\begin{fmffile}{one} 				%one.mf will be created for this feynman diagram  
  \fmfframe(1,7)(1,7){ 				%Sets dimension of Diagram
   \begin{fmfgraph*}(110,62) 			%Sets size of Diagram
    \fmfleft{i1,i2}					%Sets there to be 2 sources 
    \fmfright{o1,o2}  			 	%Sets there to be 2  outputs
    \fmflabel{$\mu^-,p_1$}{i1}			 	%Labels one of the left sources
    \fmflabel{$\mu^+,p_2$}{i2}				 %Labels one of the left sources
    \fmflabel{$e^+,p_{2'}$}{o1} 	%Labels one of the right outputs
    \fmflabel{$e^-,p_{1'}$}{o2} 	%Labels one of the right outputs
    \fmf{fermion}{i1,v1,i2} 				%Connects the sources with a vertex.
    \fmf{fermion}{o1,v2,o2} 			%Connects the outputs with a vertex.
    \fmf{scalar,label=Higgs k}{v1,v2} 		%Labels the conneting line.
   \end{fmfgraph*}
  }
\end{fmffile}
\end{center}




%%%%Section 2
\section{Assumptions and Conventions}
Throughout this derivation h and c will both be set to unity.  We will also use primed variables to denote outgoing particle/antiparticles and unprimed variables to denote incoming particles/antiparticles.  Variables with a subscript 1 will be considered particles while variables with a subscript 2 will be considered antiparticles.  We also write the momentum of the Higgs particle as $k=p_1+p_2=p_{1'}+p_{2'}=\sqrt{s}$

%%%%Section 3
\section{Getting to an Expression for the Cross Section}

We start with an expression for the rate equation
\begin{equation}
\frac{dN}{dtd\widetilde{p}_{1'}d\widetilde{p}_{2'}}=\frac{m_1m_2}{E_1E_2}\frac{(2\pi)^4\delta^4(p_1+p_2-p_{1^{'}}-p_{2^{'}})}{\Omega}\overline{\left|M\right|^2}
\end{equation}
We then can relate (1) to the differential cross section by

\begin{equation}
\frac{dN}{dt}=\frac{d\sigma v_{rel,cm}}{\Omega}
\end{equation}
So that for fermions

\begin{equation}
\frac{d\sigma}{d\widetilde{p}_{1^{'}}d\widetilde{p}_{2^{'}}}=\frac{m_1m_2}{E_1E_2v_{rel,cm}}(2\pi)^4\delta^4(p_1+p_2-p_{1^{'}}-p_{2^{'}})\overline{\left|M\right|^2}
\end{equation}
For this problem we work in the center of mass frame and note that $m_1=m_2=m_\mu$.  This leads to 4-vectors

\begin{equation}
p_1=\left(E,0,0,p_z\right)
\end{equation}

\begin{equation}
p_2=\left(E,0,0,-p_z\right)
\end{equation}

\begin{equation}
p_{1^{'}}=\left(E,E\vec{r}'\right)
\end{equation}

\begin{equation}
p_{2^{'}}=\left(E,-E\vec{r}'\right)
\end{equation}
We also know that for fermions

\begin{equation}
d\widetilde{p}_i=\frac{m_i}{E}\frac{d^3p_i}{\left(2\pi\right)^3}
\end{equation}
We also use the relation that because both the incoming masses are the same and have equal and opposite momentum (in the center of mass frame) the denominator in (3) becomes

\begin{equation}\
E_1E_2v_{rel,cm}=\sqrt{(p_1\cdot p_2)^2-m_\mu^4}=2p_zE
\end{equation}
Using (4) through (9) and substituting into (3) yields the integral for of the total cross section

\begin{equation}
\sigma=\frac{m_e^2m_\mu^2}{8\pi^2}\int d^3p_{1^{'}}d^3p_{2^{'}}\frac{\delta^4\left(p_1+p_2-p_{1^{'}}-p_{2^{'}}\right)}{p_zE^3}\overline{\left|M\right|^2}
\end{equation}

%%%%Section 4
\section{Determining the Transfer Matrix Element $\overline{\left|M\right|^2}$}
The average of the square of the transfer matrix can be written as

\begin{equation}
\overline{\left|M\right|^2}=\frac{1}{2s_{{\mu}^{-}}-1}\frac{1}{2s_{{\mu}^{+}}+1}\sum_{spins} \left|M\right|^2
\end{equation}
Because both incoming particles are fermions the prefactor before the sum is equal to $\frac{1}{4}$.  The expression for the matrix element can be read off from the Feynman diagram to be

\begin{equation}
\overline{\left|M\right|^2}=\frac{1}{4}\sum_{spins} \left|\left(\frac{-\imath gm_\mu}{2m_w}\right)\left(\overline{v}_2u_1\right)\left(\frac{1}{k^2-m_H^2+\imath\epsilon}\right)\left(\frac{-\imath gm_e}{2m_w}\right)\left(\overline{u}_{1^{'}}v_{2^{'}}\right)\right|^2
\end{equation}
After algebraic simplification


\begin{equation}
\overline{\left|M\right|^2}=\alpha\sum_{spins}\left(\overline{v}_2u_1\right)^*\left(\overline{u}_{1^{'}}v_{2^{'}}\right)^*\left(\overline{v}_2u_1\right)\left(\overline{u}_{1^{'}}v_{2^{'}}\right)
\end{equation}
where

\begin{equation}
\alpha=\frac{g^4m_\mu^2m_e^2}{64m_w^4}\left|\frac{1}{k^2-m_H^2+\imath\epsilon}\right|^2
\end{equation}
After performing the complex conjugate operations in (13) we obtain

\begin{equation}
\overline{\left|M\right|^2}=\alpha\sum_{spins} \left(\overline{u}_1v_2\right)\left(\overline{v}_{2^{'}}u_{1^{'}}\right)\left(\overline{v}_2u_1\right)\left(\overline{u}_{1^{'}}v_{2^{'}}\right)
\end{equation}
We now move the terms in parentheses around in (15) so that all the primed variables and unprimed variables are together.  We can do this because they are just numbers.  When then take the trace and use the cyclic property of the trace to obtain

\begin{equation}
\overline{\left|M\right|^2}=\alpha\text{  } Tr\left[\left(\sum_{s_1}u_1\overline{u}_1\right)\left(\sum_{s_2}v_2\overline{v}_2\right)\right]Tr\left[\left(\sum_{s_{1'}}u_{1^{'}}\overline{u}_{1^{'}}\right)\left(\sum_{s_{2'}}v_{2^{'}}\overline{v}_{2^{'}}\right)\right]
\end{equation}
We next use the completeness relation which says that

\begin{equation}
\sum_{spins} u_i\overline{u}_i=\frac{\slashed{p}_i+m}{2m} \text{  and  } \sum_{spins} v_i\overline{v}_i=\frac{\slashed{p}_i-m}{2m}
\end{equation}where $\slashed p_i$ is $\gamma^\mu p_{\mu i}$
so that we now have

\begin{equation}
\overline{\left|M\right|^2}=\alpha\text{ } Tr\left[\frac{\left(\slashed{p}_\mu+m_\mu\right)}{2m_\mu}\frac{\left(\slashed{p}_\mu-m_\mu\right)}{2m_\mu}\right]Tr\left[\frac{\left(\slashed{p}_e+m_e\right)}{2m_e}\frac{\left(\slashed{p}_e-m_e\right)}{2m_e}\right]
\end{equation}
Algebraic simplification yields
\begin{equation}
\overline{\left|M\right|^2}=\beta\text{ } Tr\left[\left(\gamma^\mu p_{\mu 1}+m_\mu\right)\left(\gamma^\nu p_{\nu 2}-m_\mu\right)\right]Tr\left[\left(\gamma^\alpha p_{\alpha 2'}+m_e\right)\left(\gamma^\beta p_{\beta 1'}-m_e\right)\right]
\end{equation}
where

\begin{equation}
\beta=\frac{\alpha}{16m_\mu^2m_e^2}=\frac{g^4}{2^{10}m_w^4}\left|\frac{1}{k^2-m_H^2+\imath\epsilon}\right|^2
\end{equation}
We next use the following properties of the trace of the gamma matrices

\begin{equation}
Tr\left[\gamma^\alpha\gamma^\beta\right]=4g^{\alpha\beta}\text{  and  }Tr\left[\gamma^\alpha\right]=0
\end{equation}
These properties allow us to drop terms with odd powers of gamma matrices.  This yields

\begin{equation}
\overline{\left|M\right|^2}=16\beta\text{ } \left(p_1p_2-m_\mu^2\right)\left(p_{1'}p_{2'}-m_e^2\right)
\end{equation}
Keeping only leading order terms in $m_e^2$ and carrying out the dot products yields

\begin{equation}
\overline{\left|M\right|^2}=\beta\text{ } 64p_z^2E^2
\end{equation}
So the expression for the total cross section becomes

\begin{equation}
\sigma=\Gamma\text{  }\int d^3p_{1^{'}}d^3p_{2^{'}}\frac{\delta^4\left(p_1+p_2-p_{1^{'}}-p_{2^{'}}\right)p_z}{E}\left|\frac{1}{k^2-m_H^2+\imath\epsilon}\right|^2
\end{equation}
where

\begin{equation}
\Gamma=\frac{g^4m_\mu^2m_e^2}{2^7\pi^2m_w^4}
\end{equation}

\begin{equation}
\sigma=\Gamma\int d^3p_{1^{'}}d^3p_{2^{'}}\frac{\delta^4\left(p_1+p_2-p_{1^{'}}-p_{2^{'}}\right)p_z}{E\left(\left(p_{1'}+p_{2'}\right)^2-m_H^2+\imath\epsilon\right)\left(\left(p_{1'}+p_{2'}\right)^2-m_H^2-\imath\epsilon\right)}
\end{equation}

\begin{equation}
\sigma=\Gamma\int d^3p_{1^{'}}d^3p_{2^{'}}\frac{\delta^4\left(p_1+p_2-p_{1^{'}}-p_{2^{'}}\right)p_z}{E\left(4E^2-m_H^2\right)^2}
\end{equation}
Carrying out the integral over the spatial portion of the delta function picks out one of the $d^3p$ integration elements and leaves the integral with a delta function in the zeroth component (i.e. the energy).

\begin{equation}
\sigma=\Gamma\int 4\pi E^2\frac{p_z\delta^4\left(2E-\sqrt{s}\right)}{E\left(4E^2-m_H^2\right)^2}dE
\end{equation}
Simplification and integration over the remaining delta function yields

\begin{equation}
\sigma=\frac{g^4}{2^7\pi}\frac{m_e^2m_\mu^2}{m_w^4}\frac{\sqrt{s\left(s-4m_\mu^2\right)}}{\left(s-m_H^2\right)^2}\text{\hspace{1cm}Q.E.D.}
\end{equation}

%%%%Section 5
\section{Conclusions}
From the final expression of the cross section for Higgs production at a muon collider there is a factor of $m_\mu^2/m_e^2$ when compared to the cross section for an electron positron collider.  This would be the main advantage of muon collider.  However, muons are unstable particles and building a collider which produces, collects, accelerates, and collides them in their lifetime is problematic.  

\end{document}




















