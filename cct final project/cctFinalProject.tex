\documentclass{article}
\usepackage{amsmath,amssymb}
\usepackage[margin=1in]{geometry}
\title{Teaching Portfolio}
\author{Brad Schoenrock}
\date{}
\begin{document}
\maketitle
\Large
\tableofcontents

\newpage
\section{Curriculum Vitae}
\subsection{Education}
\hspace{0.5cm}PhD seeking Graduate Student MSU 2010-present
Masters Michigan State University 2012
REU at Cornell University 2009
B.S. physics and mathematics with minors in chemistry and computer science Northern Michigan University 2006-2010
\subsection{Research experience}
\hspace{0.5cm}Search for tZ final state in ATLAS, Advisor: Dr. Reinhard Schwienhorst
Calculation of diboson cross sections at the LHC, Advisor: Dr. Reinhard Schwienhorst
Single top quark cross section measurement in the t-channel at the high-luminosity LHC, Snowmass 2013, Advisor: Dr. Reinhard Schwienhorst
Creating the planetarium show "Relics of the Big Bang" for ATLAS outreach, Advisor: Dr. Reinhard Schwienhorst
Fitting reaction rates for astrophysically relevant nuclear processes, Advisor: Dr. Hendrik Schatz. 
Assembly of neutron detectors for a measurement of the electric form factor of the neutron, Advisor: Dr. William Tireman
Soil and sediment sampling project (Searching for Naturally Occurring Radioactive Materials) in Marquette MI, USA, Advisor: Dr. William Tireman
\subsection{Publications and Papers}
\hspace{0.5cm}Single top quark cross section measurement in the t-channel at the high-luminosity LHC
Brad Schoenrock, Elizabeth Drueke (Michigan State U.), Barbara Alvarez Gonzalez (Michigan State U. \& Cantabria Inst. of Phys.), Reinhard Schwienhorst (Michigan State U.). Aug 28, 2013. 16 pp.
e-Print: arXiv:1308.6307
Snowmass 2013 Top quark working group report
Top Quark Working Group Collaboration (K. Agashe (Convener) et al.). Nov 8, 2013. 54 pp.
e-Print: arXiv:1311.2028 
Searches for resonances in the tb and tc final states at the high-luminosity LHC
with contributions from the ATLAS Collaboration (Elizabeth Drueke et al.). Sep 26, 2013.
e-Print: arXiv:1309.7043
\subsection{Presentations}
\hspace{0.5cm}"Single top quark cross section measurement in the t-channel at the high-luminosity LHC" Snowmass proceedings, 2013
"Relics of the Big Bang-Live talks" Abrams planetarium, 2011
"Developing Automatic Test Electronics for Measuring the Performance of SRF Cavities" at Argonne National Laboratory’s Symposium for Undergraduates, 2009
"Neutron Detector Assembly for a High Precision Measurement of the Electric Form Factor of the Neutron" at Argonne National Laboratory’s Symposium for Undergraduates, 2008
\subsection{Languages}
\hspace{0.5cm}I am familiar with C++, root, LaTex, java, labview, MATLAB, some bash, some Visual Basic, and some assembly code. 
\subsection{Awards and Acknowledgements}
\hspace{0.5cm}AAPT Best Graduate TA, 2012-2013
Graduated Suma Cum Laude NMU, 2010
Honors Graduate, Northern Michigan University, 2010
Outstanding physics undergraduate Northern Michigan University, 2010
\subsection{Job experience}
\hspace{0.5cm}Research assistant MSU high energy physics group, 2010-present
Teaching Assistant, MSU physics department, Introductory Physics Laboratory 1, 2010
Research assistant National Superconducting Cyclotron Laboratory, 2010
Walk in tutor at NMU, Marquette MI, 2008-2010
Research assistant physics dept. at NMU, Marquette MI, 2008
Individual tutoring for Student Support Services (SSS) at NMU, 2007-2010
Walk in tutor for SSS, Marquette MI, 2007-2010
Study group tutor for SSS covering introductory physics 1 \& 2, Marquette MI, 2007-2010
Lab setups for the physics dept. at NMU, Marquette MI, 2007-2010
\newpage


\section{Teaching Philosophy}
My goal in teaching would be to implement methods that allow for all learning types to have opportunities to succeed. I would do this by ensuring plenty of demonstrations of physical principles at work and getting the students involved in these demonstrations. Part of what makes physics unique is that what we teach in the classroom will reflect reality and can be readily tested with the simplest of setups. This means there are many ways to incorporate demonstrations into lectures. It is my experience that these techniques work best in small classrooms(such as the ones I learned in at Elcho High School and Northern Michigan University), where the student has a chance to be engaged and get to know the professor, rather than being able to hide in the corner in a class of a hundred students. Through these active learning techniques, small classrooms, and proper curriculum development students should be able to learn science the way science is really done, by direct observation of these demonstrations in class, questioning what they saw, coming up with ideas for how to explain what they saw, testing their predictions, and coming to conclusions about the validity of the hypothesis they themselves came up with. In utilizing this methodology students will not only understand what is covered in class better, but will be better prepared for more advanced class work and future employment. 
Larger classrooms are unfortunately a necessity in modern academia, and need to be handled delicately. Many of the techniques that I have found do not "scale up" very well into a large classroom, often coming across as a gimmick to the students who know that it isn't the focus of that lecture and are often ignored by the students. I have found that the best way to combat this is to give small 0-1-2 style graded assignments every day in every class. This works in two ways. First and foremost it ensures that students are coming to class and it offers a tool for formative assessment of the students in a variety of ways. If the small assignment changes format frequently it keeps the students engaged and interested to see what they will be doing today, and in so doing, forcing your students to work outside their comfort zone becomes crucial for this methodology. 
\newpage


\section{Teaching Experiences}
My teaching experience started young with tutoring math and science to my classmates in high school and those younger than me in middle school. My tutoring experience continued throughout college where I tutored students with disabilities with Student Support Services, ran study groups, and did walk in tutoring for All Campus Tutoring. In graduate school I was a TA for introductory physics libratory where I was responsible for administering quizzes and grading lab reports. Continuing with teaching I have begun the certification of college teaching program at MSU.


\section{Professional Development}
As a tutor at Northern Michigan University for Student Support Services I underwent formal tutor training and weekly meetings where we reviewed tutoring methodology. The SSS Tutor Training program is certified by the College Reading and Learning Center Association (CRLA). 
I served on the committee for graduate student curriculum development in the department of Physics and Astronomy where we discussed the feasibility of merging graduate courses to expedite graduate student coursework. 
I have also begun the certification of college teaching program at MSU which includes a formal class on teaching college science, a two day seminar on teaching college science, and a mentored teaching experience. As part of this certificate i created a series of introductory videos to our intro physics lab for non-majors, evaluating the educational objectives and tailoring the videos to achieve those objectives. Also in the course of completing this certificate I wrote a syllabus and a lesson plan for a modern physics course (attached). 
\newpage


\section{Teaching Evaluations}
\hspace{0.5cm}The most adequate teaching evaluations that I can report on are the student evaluations from my semester of TA work. Fall of 2010 I was the TA for PHY 251 which is introductory lab for non-majors at MSU. 
The Evaluations were broken up into two sections, a short answer and free response. The free response section had the following questions. 
\begin{enumerate}
\item What helped you learn?
\item What hindered (got in the way) of your learning? 
\item What questions do you have that you would like answered?
\item What changes would you suggest to improve your learning?
\item Would you recommend this course to other students? Explain.
\item Other comments.
\end{enumerate}
\subsection{compatibility with Code of Teaching Responsibility}
\hspace{0.5cm}One of the questions was "The laboratory was conducted in a manner consistent with the Code of Teaching Responsibility". This was ranked on a scale of 1-5 (1=strongly agree, 5=strongly disagree). The results were a mean of $\mu=1.68$ with a standard deviation of $\sigma=0.66$. so a perfect score is just barely out of the one standard deviation range. One reason for this is that many of the students didn't like the fact that the course was graded on a curve. 
\subsection{Grade the TA}
\hspace{0.5cm}Another (perhaps more relevant) question was on a 4.0 scale grade your TA. The results were a mean of $\mu=3.44$ and a standard deviation of $\sigma=0.69$. As we teach our students, that measurement is compatible with a 4.0. 
\newpage
I have also compiled the following responses that contained the words TA, Brad, teacher, explain, questions, and examples. I chose those words to ensure the responses here were related to my work as a TA. 
\subsection{What helped you learn?}
\hspace{0.5cm}The TA
The TA going over concepts before lab
Having the TA work through things with me
TA explaining the content
The TA
Explanation of questions
Being able to ask questions
Brad, our TA
The TA going through the examples
Asking for help
Explanations in the beginning of lab helped clarify the manual
Brad the TA was very helpful when we had questions and he explained everything really well
Watching the TA do examples with everyone watching
My TA was very helpful and did a great job
The lab manual and the TA
The TA was very helpful when he explained things and answered questions. 
\subsection{What changes would you suggest to improve your learning?}
\hspace{0.5cm}I don't know, letting the TA's help us more
Make the teacher more available
\subsection{Other comments}
\hspace{0.5cm}Brad did a great job helping us understand the experiments
The TA, Brad, was great. He was willing to answer my questions and clarified when i didn't understand something.
Good job :)
\newpage


\section{Mentored Teaching Project Proposal}
\subsection{Course Description}
\hspace{0.5cm}This project will focus on PHY 251 introductory lab for non-majors. These students are typically freshmen/sophomore students who are not in physics but are in the sciences and/or premed students. With this there is some expectation for the demonstration of critical thinking skills resembling the scientific method although they are not designed for this purpose. PHY 251 is a once a week class for 1 hour and 50 minutes in which students take a quiz, listen to a short introduction to the lab that week, collect data, analyze the data, and fill out a worksheet on the material that was covered that week. Over the course of the semester students are expected to become familiar with data taking strategies, uncertainties, creating histograms and plots, discussing significance of results, and to observe the general physics principals that are taught in the lecture class for non-majors (PHY 231). The instructor for PHY 251 in the fall 2012 semester will be Jim Linnemann, and in the spring 2013 will be Joey Huston. I will also be working with Gerd Kortemeyer and Richard Halstein for advice and guidance. My graduate advisor is Reinhard Schwienhorst.
\subsection{Problem}
\hspace{0.5cm}For this project I would like to address this question of preparedness. Many students arrive in lab with skewed understanding of that week’s lab procedure and what they will be expected to do. This leads to students running out of time in the 1 hour 50 minute lab and this can be frustrating for the students as well as the instructors. To a certain extent this problem is because they don't read the manual before they arrive and are therefore underprepared. The argument could be made that if they didn't want to read the manual, then they will not participate in other exercises before lab, but I would argue these videos are not focused on those students. The target audience is the student who is trying hard but isn't getting the help from the lab manual or their TA's and instructors. I hope that creating introductory videos will help these students who, through no fault of their own, are having a hard time with the course. 
\subsection{Research Question}
There are three complementary goals to this project. 
\begin{enumerate}
\item Will student preparedness increase if they are given videos to help them prepare? 
\item Will student understanding improve if they are given videos to help them prepare? 
\item Will student opinions on the course change if they are given videos to help them prepare? 
\end{enumerate}
\subsection{Hypothesis}
\hspace{0.5cm}I believe that one factor for why the students are underprepared for lab is because the lab manual either isn't giving a clear understanding of what the equipment is, the manual isn't clear on how to use it, or perhaps both. This is what the TA is supposed to be for in each section (to clear up a misunderstandings introduced by the manual) but this is not always successful. In this case my hope is that these videos will serve as an augmented introduction to the lab to allow students to become familiar with the equipment and procedures independently of their TA's introduction and the lab manual. I also think that these videos have the potential to be entertaining, and therefore more engaging, which might lead to higher participation. 
\subsection{Procedure}
\hspace{0.5cm}During the fall semester I will record myself doing an intro to each lab, with a demonstration of the lab equipment properly set up, pointing out some common mistakes that are often made. These videos will be hosted online for viewing for the spring semester, when they will be implemented and ready to use and assess. 
My plan is to have students, on a voluntary basis, watch the videos before they get into lab. Watching these videos will not be linked into student grades (except in that watching the videos might help prepare them better) to ensure they are received by the professors of the physics department. That way we will not change the format of the lab section. I will poll students, TAs and the professors of 251 and 252 (the second semester of introductory physics lab) to determine if they are being watched, if they were thought to be helpful, and if they were thought to help prepare students for lab. Success will be determined by positive feedback from all involved.
\newpage


\section{CCT reflections}
\subsection{Disciplinary Teaching Strategies}
\hspace{0.5cm}Disciplinary teaching strategies are important in every field because every field is unique. While it is important to have general teaching strategies that work for every discipline (such as formative assessment in multiple formats) no technique will be all encompassing. For this reason it is imperative that we consider our teaching strategies not separate from, but within the context of our field (in my case, physics). 
This Core Competency was fulfilled through the SME 870 class "Teaching College Science". This course addressed this competency in many ways. we discussed many broad teaching strategies such as backward design, blooms taxonomy, and the importance of group work but we were also able to address concerns relevant to teaching science such as how to pick a textbook that isn't too equation heavy, creation of a syllabus in our field, and discussion on educational objectives in a laboratory setting. 
Picking a good textbook can be extremely important to students learning. As a student I have seen examples of both good and bad textbooks in classes. A good textbook can be an excellent supplement to a class that is useful to a teacher because it does not require face to face time with the student, but the student can still learn much of the important information. While this is useful to the teacher it can be much harder for a student to learn from a textbook, and therefore the textbook must be chosen carefully to reflect what the teacher wants to impart on the students. This can cause a trap for those who have written their own textbook however because they can be tempted to use their own textbook in their class. This is a problem because the students go to the textbook for an alternate explanation of the material and if the lectures are a repetition of the book then this isn't useful. 
The creation of a syllabus is vital to understand how the class is going to flow and will help decide which are the more important topics to be covered. While the syllabus has other goals (to inform on grading etc...) it allows the student to see where the class is going. It also forces the teacher to think about what they are going to teach instead of just showing up and talking. An example syllabus is attached.
The discussions on laboratory objectives were quite useful in the implementation of my mentored teaching experience. It was enlightening to see how many professors in the physics department consider the introductory labs for non majors to be scientific inquiry style labs. Upon further analysis it is very clear that they are not. They are very clearly confirmation labs. These labs are designed to have students follow directions to practice lab techniques and confirm the accuracy of physical laws and theories in order to validate their understanding. Once that was understood it was clear how to proceed. I had to make the videos to the goals of a confirmation lab instead of for a scientific inquiry style lab. If I had made the videos in the way that professors envisioned there would be a sort of cognitive dissonance for the student between the videos goals and the goals of the lab. This certainly would not be useful for the students, or for the teaching assistants who will teach using these videos in the future. 
\subsection{Creating Effective Learning Environments}
\hspace{0.5cm}Creating effecting learning environments is important for all students. This isn't necessarily an easy thing to do because all students learn differently. the challenge here is to understand how your class learns as a whole and tailoring your teaching style to accommodate their various learning styles, communication skills, motivation ect.... The problem comes in changing your teaching style. Most teachers have a hard time changing how they teach but it can be vital to having a successful semester. 
To fulfill this core competency I attended the Certification in college teaching institute May 10-11. 2012 where we heard the director of the writing center, Trixie Smith, discuss "Writing Across the Curriculum". Trixie focused on how to incorporate writing into classes that traditionally don't utilize this important aspect of learning. This is part of two complementary topics. The first is how we collaborate as teachers across classes to provide a cohesive learning environment for students seeking a broad education (as all students should.) English isn't just a subject to be taught by the English department just like science shouldn't be solely relegated to science professors. There is certainly not only room but a necessity for physics students to understand how to write effectively and for English students to understand how to disseminate a scientific argument. The second is teaching how to effectively communicate the information both taught in and learned out of the classroom. 
Effective communication is important in physics because we don't have enough of it. there is a lot of exciting research in all scientific fields but the articles describing these phenomena don't do the subject justice because they are dry reading, unnecessarily convoluted, and written by people who never received the necessary education to be an effective communicator. The tides of change are coming in this as science writing has become a viable profession for magazines and newspapers but there still has been little change in scholarly articles. It seems the most effective remedy to the situation is to teach students the communication skills they need to be scientists. 
Collaboration across disciplines (in both directions) is perhaps the best way of going about this change. English majors should read not just Shakespeare and Dostoyevsky but perhaps "Ideas and Opinions" by Albert Einstein and "The Universe in a Nutshell" by Stephen Hawking to analyze how effectively (or not) non-writers are communicating their work. In the course of teaching physics we need to emphasize communication better. There are several ways we can accomplish this. Firstly writing assessments need to be a probe for deeper understanding. This is a level of assessment that is lacking in the sciences. We often assign problem set style homework but this only assesses the lower levels of learning (remembering, understanding and applying.) few assessments we make ask the students to analyze different situations or compare the similarities between two physical situations while virtually none ask for evaluation or creation. This is a good place to incorporate writing into our day to day physics classes. Giving paragraph questions in exams and/or routinely asking students to write short pieces describing the lesson of the day and having them do a peer review on those papers forces students to evaluate the information given. this will not only help students understand the course material better but will also teach them to communicate more effectively in their day to day lives as they encounter confusing and complex topics. More importantly this changing of the traditional problem set paradigm will lower the barrier between the scientific and the social disciplines allowing for students who aren't naturally problem solvers to still take something away from the course while allowing those who are problem solvers to continue with problem sets. Ensuring that all students have something that they can succeed at means that every learning style will leave with a sense of accomplishment and I hope this will leave the students more confident and more competent in both in the sciences and in the humanities.
\subsection{Technology in the Classroom}
\hspace{0.5cm}Technology in the classroom has quite a lot of debate around it regarding everything including its format, implementation, effectiveness, the list goes on. I feel this is largely because of its relative youth and rapid evolution compared to more traditional teaching methodology. I’ve seen technology incorporated into classes very well with simple non invasive strategies such as clicker questions and open computer note taking policies. but I’ve also seen technology used in terrible ways such as the case with some ill-prepared notes used as a replacement for well formatted PowerPoint slides or well managed blackboard work that leaves the students unable to follow the discussion due to poor formatting and quick pacing. In this way it is a double edged sword of sorts and needs to be handled with care so as to add to the discussion of the topic at hand without detracting time and effort explaining it or using technology for the novelty. 
To fulfill this core competency I attended the Certification in college teaching institute May 10-11 2012 where we heard from a panel of MSU professors including Carol Wilson-Duffy, Kathy Doig and Stephen Thomas. They discussed several ways that technology can be introduced to the classroom. 
One of the more interesting methods was actually out of the classroom applications by assessing students using tools integrated into blackboard or LON-CAPA. I think these tools have the potential to be quite effective if used properly. One advantage of such tools is computerized grading. This allows students to get instant feedback on their mistakes without taking the instructor's time. The drawback to this is that the feedback may not be useful, and if the student has any questions on why they didn't get a problem right they have nobody to ask. This can lead to a student becoming frustrated and quitting which is not the reaction we want from struggling students. These tools are better used to submit weekly writing assignments, or to answer open ended questions. The professor can then grade these assessments with appropriate feedback to give the student a better picture of what has been misunderstood. Unfortunately in my experience this is not how these systems have been used. Instead they are used as graders for multiple choice questions. Used properly these tools could provide extra projects for students who find class work too easy while helping along students who find the class work too difficult. 
I have also seen these tools used to provide students with additional readings and to post the syllabus, but this isn't what these tools were designed for and this functionality could be better conveyed through a course website. 
Another topic of discussion was integrating technology for student assessment. The first way I have seen this done is having students log into a tool such as blackboard and answering questions during class through this tool. The problem is that computers provide their own distractions and students lose interest in class. Perhaps a better way of doing this is through clickers. Clickers provide all the same formative assessment power without the distractions a computer can pose. This allows teachers to assess how the students understand the material presented as it is presented. Once this is accomplished you can review something that students aren't quite getting or move past something that they clearly understand thereby increasing efficiency of class time. 
Another use of technology is the use of out of class quizzes used to save class time. Quizzes are generally useful for summative assessment each week, but this takes time and doesn't give immediate feedback. Online quizzes also have the drawback of cheating. Groups of students can sit in a room together and ensure they all get perfect scores by giving each other answers to the questions they get. Online classes suffer from the same issues that online quizzes have. This is a problem to which I have seen no studies that measure the rate or significance of these behaviors or any good way to prevent them if they were detrimental in some way. In this same way many technologies used for teaching have been explained as being beneficial but never measured to be effective. They proliferate because they are well liked and one would hope they are well liked because they are effective, but that may or may not be the case. In the interim we use technology as an attempt to help students learn, and we believe it does. 
\subsection{Understanding the University Context}
\hspace{0.5cm}Understanding the University context is relevant for all professors to understand their rights and responsibilities to their institutions as well as their institutions rights and responsibilities to them. This means that the professor needs to have a professional attitude toward teaching by continuing to develop their teaching strategy and skills while the university needs to provide and environment in which this is possible by providing professional development opportunities such as workshops to develop teaching strategies. It is also important here to understand your relationship to your class as one that is both instructive to the students and conveying a sense of professionalism to the university. 
To fulfill this core competency I attended the Certification in college teaching institute May 10-11 2012 where we heard from Rique Campa and Judith Stoddart who discussed "Developing Your Teaching Presence and Philosophy". Here they highlighted the fact that teaching begins long before you step into the classroom. It begins with a plan where you emphasize the points that you think are important by developing a well thought out curriculum and picking a textbook that will suit the needs of the course that you will lay out. Then while thinking about the style you will teach the course the classroom needs to be considered. If you are teaching an upper level course to 10 students you will treat the material much differently than you would to a room of 300 students. Another issue addressed was the reuse of testing material. As an instructor you have a duty to honestly assess student performance and the reuse of testing material often leads to cheating by having old materials passed down from older generations of students. Once this happens the quizzes and tests don't faithfully represent the student’s ability to learn new material, only their ability to copy what is given effectively lowering the level of assessment from a higher level analyzing or evaluation to simple remembering. Ensuring that you take these duties seriously is not just an obligation to your students but to the university as well to ensure the university is seen in a professional way. 
Also in the fulfillment of this competency SME 870 "Teaching College Science" covered some interesting topics when Robert Caldwell (MSU ombudsman) discussed how professors interact with the university and with students. Some of what was discussed should be common sense regarding treatment of students. There were also some issues that I hadn't thought of until they were brought up. 
Instructors are allowed to change their syllabi during the semester for any reason. This means that they can amend their schedule, attendance policy, grading criteria, and academic dishonesty policies at their will. I have never known a professor to change their dishonesty policy but it is good to know that the syllabus isn't a binding contract. It makes sense that it isn't because if it were students could game the system to their advantage using the syllabus as a shield from cheating if it suited them. This is useful because it means if I would like to try a new assessment method mid semester, or crack down on a particularly bad case of academic dishonesty, I can do so without worries. 
Another interesting point is that a professor cannot permanently dismiss a student from the class if the student commits an act of academic misconduct. The reason is that a student can contest this allegation and if they are absolved of the act then they are still enrolled in the course and have every right to attend class and have their work graded in an impartial way. This brings up another point, if a student has a history of academic dishonesty, as a grader you are not allowed to put extra scrutiny into their work. It might feel like you should keep extra watch on this student but this introduces a selection bias. If you are looking for incidents of academic dishonesty in this student you are more likely to find them whether they exist or not. 
A more frustrating fact about the reporting of academic dishonesty is that students who have cheated in your class can drop the class. Once they drop the class a professor is unable to file a report of academic dishonesty. This means that a student who is accused of cheating who knows they cheated can drop the course before anything can be done about it. This means that if a professor wants to take a case of academic dishonesty to the university (as the university requires) they need to do so without talking to the student first. This leaves me unsettled because I believe you should talk to the person before going through official channels. I believe this is a broken system and needs review. 
Lastly professors have a responsibility to take cases of academic dishonesty up to the university. This is the only way to catch repeat offenders in the university setting. If professors don't step up and handle students each time an incident comes up then students can continue the same habits without consequences and will continue to toe the line of academic dishonesty in future course work. 
\subsection{Assessment of Learning}
\hspace{0.5cm}Assessing student learning is important for a teacher on a day to day basis. It is the feedback a teacher gets on their ability to impart learning on their students. One problem that can arise from the lack of assessment is that many teachers are happy to go on teaching without concern for whether they are imparting learning to their students. Measuring the degree of success in student learning is vital in avoiding failure. 
To fulfill this core competency I attended the Certification in college teaching institute May 10-11 2012 where we heard from Tammy Long. Also in the fulfillment of this competency I participated in a mentored teaching experience. 
At the institute many interesting methods for student assessment were presented. Throughout the conference formative assessment was emphasized as a tool to assess if students understood the material we think we are teaching. One popular method of doing this is to use clickers instead of hand raising techniques. The advantage of this is that students must answer the question posed without any knowledge of what other students are answering. In this way you get a more clear representation of the classes understanding individually because you can see trends of students who answer correctly or incorrectly. You can also assess the class as a whole rather than an assessment of the few confident students who raise their hands while everybody follows along. This technique also forces students to face their misunderstandings head on. Once they answer incorrectly they know they must pay better attention or it will be reflected in their performance. The downside to clicker questions is that they are multiple choice meaning it retains all the flaws therein. 
In the course of my mentored teaching experience I came to realize some of the shortcomings of using grades as a metric to evaluate the success or failure of certain teaching methods. One flaw is that introductory physics lab is graded by a dozen different TAs leading to differences in scores that arise not from student preparedness but from the grader. Another issue is that student preparedness can arise from sources not in study; the fact that these videos are not used in isolation from other course materials (and shouldn't be) means they cannot be assessed alone. Utilizing a previous semester as a control would be problematic because of different TAs giving different grades, different majors among the students attending between semesters, and TAs will renormalize their grading to give the same distribution in grades regardless of how well students do (any improvement on the part of the students will be reflected in harder grading among the TAs). For these reasons i have opted for a more informal assessment of the impact of this project. 
After speaking with several students in the class, graduate TAs, and professors who have overseen the implementation of this project there are several conclusions that can be drawn. Students found the videos useful because they were short, informative, and engaging. Often the students wanted to see a more thorough procedure outlined in the videos which I deferred to the lab manual. This desire reflects the fact that students wanted to see a replacement for the lab manual, which was neither the original extent nor design of this experiment. TAs found a better sense of confidence in the students in topics covered in the videos, but found that the overall number of questions and help needed didn't noticeably decrease. This might indicate that the students understood the material covered but found new questions to ask. The professors found that because these videos existed for the first semester of labs (but not the second semester) students complained because they didn't have the videos available for the second semester. All in all I think these are indications that the videos were well liked and were helpful teaching tools. I see this positive outcome as a lead in to perhaps more thorough introductory videos or interactive introductions for the students with more extensive study in the future. 
\end{document}


