\documentclass{article} 
\usepackage{graphicx}
\usepackage{amsmath}

\begin{document}
\pagestyle{empty}


\begin{center}
\bf PHY 251 Week 1 worksheet
\end{center}

\section*{\normalsize Uncertainties in Measurement}

This exercise is to help you understand how uncertainties are estimated and utilized. See appendix A: Dealing With Uncertianty for more information. 

\vspace*{0.25cm}

1. Measure the width of a piece of paper with a ruler and estimate the uncertainty of your measurement.  A standard piece of paper is $21.59 \pm 0.01 cm$ wide. Is your measurement consistent with this?

\vspace*{1cm}

2. Now crunch the piece of paper into the ball and measure the ball's diameterwith a ruler and estimate the new uncertainty. Is your measurement consistent with your partner's?

\vspace*{1cm}

3. Discuss the factors that go into estimating the uncertainty of the first and second measurement. What important difference is present for the second measurement?

\vspace*{5mm}

\section*{\normalsize Error Propagation}

This exercise is to help you understand how errors from measurements propagate as the numbers are used in equations. See appendix A: Dealing With Uncertianty for more information. 

\vspace*{0.25cm}

4. If one student measures the distance from Detroit to Lansing to be 90 $\pm$ 5 miles, and another student measures the distance from Lansing to Grand Rapids to be 70 $\pm$ 10 miles, then what is the total distance from Detroit to Grand Rapids. Be sure to include uncertainty and units.

\vspace*{1cm}

5. What is the difference in the two distances?  Be sure to include uncertainty and units.

\vspace*{1cm}

6. Measure the width and length of a piece of paper with a ruler and estimate the uncertainty of each measurement. Find the area of the paper and it's associated uncertainty with units.

\vspace*{1cm}

\section*{\normalsize Graphs}
This exercise is to help you understand how graphs are read and interpreted.

\vspace*{0.25cm}

7. A student measures the position of a moving car as a function of time and plots the data below. The equation describing the car's motion is $x = x_0 + v t$, where $x$ is the position of the car (in meters) at time $t$ (in seconds), $x_0$ is the initial position of the car, and $v$ is the velocity of the car. The student uses Kaliedagraph to compute a best-fit line for this data, shown on the graph.  Does the car move with constant velocity?

\vspace*{1cm}

8. What is the velocity of the car? Be sure to include units and uncertainty!

\vspace*{1cm}

9. Another student measures the position of the same car but much further away (so the uncertainty of his measurements is much higher!) and makes a plot of his data similar to the first student, shown below. What velocity did the second student measure? Is his velocity consistent with the first student's measurement of velocity?

\vspace*{1cm}





\end{document}