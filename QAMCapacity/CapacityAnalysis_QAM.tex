\documentclass{article}
\usepackage{amsmath,amssymb,url}
\usepackage{graphicx}
\usepackage[table,x11names]{xcolor}
\usepackage{float}



\author{Brad Schoenrock\\Video Operations Engineering\\Charter Communications\\Greenwood Village, CO}
\title{Stitcher Capacity Analysis:\\Charter Internal Note.\\Draft Version 0.0.0}
\date{July. 2019}
\begin{document}
\maketitle
\newpage

\tableofcontents
\newpage

\section{Introduction}
\label{SECTION-Introduction}

Stuff and things... 

Spec-Guide QAM bandwidth was initially a conern brought up as a possible cause of Guide Unavailable, Failure to Tune messages, and a potentially call volume generating problem. Spec-Guide QAM bandwidth is a limited resource, which scales with the number of customers, so the capacity of the Spec-Guide QAMs were also a concern to be investigated. The architecture of the platform dictates that each service group gets exactly two Spec-Guide QAMs to serve it, with no more or less. This creates a situation where the RF bandwidth of the QAMs dictate the maximum number of customers a service group can support. Since the QAMs themselves can't output more bandwidth (They are fixed devices that don't lend themselves to upgrades) and our architecture dictates that we can't add more than two QAMs per service group, this means that the only lever that is available to resolve such issues is managing the size and number of our service groups. We (ISP) does this routinely with service group splits and merges. When service groups get too large they are split into two different service groups, and when small service groups are near each other they are collapsed into one service group. 

On a service group by service group level the QAM bandwidth is managed on the CSM by the QAM Resource Manager (QRM). The QRM is set up through configuration files, topology.xml and programs.xml. These config files define the total bandwidth output capability of the QAM for the QRM, the output frequency of the QAM, and the bandwidth expected for each type of stream (SD vs. HD, etc...) in both steady state operations as well as a maximum bandwidth usage. In this way we can monitor estimated QAM usage and ensure there aren't any overruns. As overruns would be approached, the QRM uses smartmuxing in order to serve extra sessions when current sessions are not using up their allotted bandwidth. 



\section{QAM Bandwidth Availability \& Analysis}
\label{SECTION-QAMCapacity}

Stuff and things... 

Each QAM has 38.8MBps of bandwidth, 37.5MBps is usable for delivery of Spec-Guide sessions. With two QAMs per service group this means each service group has 75MBps of bandwidth for delivery of Spec-Guide sessions. 

The QRM is configured based on actual session usage, a sample configuration can be seen in Appendix~/ref{APPENDIX-QRMConfig}. The highest definition mpeg streams allocate 2.5MBps steady state, and a peak of 6MBps of RF bandwidth. By monitoring the throughput of the APEX by eye in prime time these settings are matching usage. Average session bandwidth is between 2-2.5MBps which is below 2.5 since not every session is HD. 

In order to define a functional capacity we need to ensure enough bandwidth to serve another session. For this reason 6MBps of bandwidth should be reserved on each QAM. The remaining 63MBps of QAM bandwidth on the service group can then serve 26 HD sessions at the configured 2.5MBps each. Maximum concurancy rate of our customer base was measured to be 4\% in stitcher capacity analysis, and while that would fluctuate more on a service group basis, it still serves as a baseline for system usage. This leads to a capacity calculation of: 

$$Capacity=\frac{QAM_{BWTotal}-2*MAX_{BWSess}}{SS_{BW}*Concurancy}$$ 

Where $Capacity$ is the number of customers a service group can handle defined by QAM bandwidth, $QAM_{BWTotal}$ is the total usable QAM bandwidth available on the service group, $MAX_{BWSess}$ is the max bandwidth used for a session (typically during session set up), $SS_{BW}$ is the steady state bandwidth usage, and $Concurancy$ is the concurancy rate of users in the service group. 

This calculation leads to a capacity of approximatly 720 customers per service group. There are several reasons why this is a low estimate. 

\begin{enumerate}
\item The average usage is lower than an HD stream would dictate. This means more sessions can be allocated to a QAM than if they were all HD streams
\item The allocation of keeping max bandwidth in reserve per QAM is overkill. Not every session will use the max bandwidth on setup, so it is possible that less can be reserved for new sessions. 
\item A 4\% concurancy rate is on the high end of measured usages. Fluctuations to that value would be possible, but the majority of cases would be lower, and more users could be supported at lower concurancy rates. 
\item The QRM uses smartmuxing which ``Recycles'' unused bandwidth in order to serve out more sessions than this analysis would indicate. 
\end{enumerate}

These factors provide significant leeway in capacity concerns. Operating at 720 customers per service group would be operationally ideal, but these factors would easily allow us to reach 800-1000 customers per service group without concerns to the customer experience. If we were to reach 1200-1500 customers per service group then the customer experience would begin to suffer due to Spec-Guide QAM bandwidth. 

Customers per service group was measured through the CSM logs (but is also tracked in other ways by other groups) in order to determine if any service groups were over the relevant thresholds. Table~\ref{TABLE-HighSGUse} lists service groups to watch. It is noteworthy that the average number of customer per service group is INSERTNUMBERHERE, (which is BLAHBLAHBLAH lower than the first threshold) so overall we have pleanty of QAM bandwidth to go around so long as it is properly allocated. 



\section{Conclusion}
\label{SECTION-Conclusion}

Stuff and things... 

If sessions begin to overrun the Spec-Guide QAM bandwidth beyond what smartmuxing could handle, the customer experience would be affected in a few ways. First the QRM will send a few frames of lower quality for sessions in order to limit the bandwidth usage. Next, users would experience pixelation of the guide interface, delayed frames, and/or droped frames in order to maintain the integrety of the Spec-Guide QAM. As a last resort the QRM will deny new sessions outright. The QRM will not allow sessions to be sent to a QAM once the bandwidth has been entirly allocated. All of these behaviours would be recorded in CSM logs. 



Stuff and things... 

There are no significant Spec-Guide QAM bandwidth related concerns. 

The best way to ensure we are not overrunning our QAM bandwidth is to limit the number of customers per service group, and perform service group collapses and splits as needed. 

Direct measurements of RF bandwidth in an automated fasion would be time consuming, costly, and unnessicary. The number of customers per service group serves the same purpose in this case, and requires no further monitoring. 


\newpage

\appendix

\section{Appendix A\: Example QRM Configuration}
\label{APPENDIX-QRMConfig}

Stuff and things... 
\newline
\newline

$<$resolution height="480"$>$

$<$mpeg2$>$

$<$bitrateProfile name="Default\_SD"$>$1000000$<$/bitrateProfile$>$

$<$bitrateProfile name="Low\_SD"$>$1000000$<$/bitrateProfile$>$

$<$bitrateProfile name="MediumLow\_SD"$>$1200000$<$/bitrateProfile$>$

$<$bitrateProfile name="Medium\_SD"$>$1500000$<$/bitrateProfile$>$

$<$bitrateProfile name="High\_SD"$>$2000000$<$/bitrateProfile$>$

$<$bitrate$>$4000000$<$/bitrate$>$

$<$quant$>$3$<$/quant$>$

$<$aquant$>$4$<$/aquant$>$

$<$txtquant$>$2$<$/txtquant$>$

$<$/mpeg2$>$

$<$h264$>$

$<$bitrate$>$4000000$<$/bitrate$>$

$<$quant$>$14$<$/quant$>$

$<$aquant$>$14$<$/aquant$>$

$<$/h264$>$
\newline
\newline

$<$resolution height="720"$>$

$<$mpeg2$>$

$<$bitrateProfile name="Default\_HD"$>$1600000$<$/bitrateProfile$>$

$<$bitrateProfile name="Low\_HD"$>$1600000$<$/bitrateProfile$>$

$<$bitrateProfile name="MediumLow\_HD"$>$1800000$<$/bitrateProfile$>$

$<$bitrateProfile name="Medium\_HD"$>$2000000$<$/bitrateProfile$>$

$<$bitrateProfile name="High\_HD"$>$2500000$<$/bitrateProfile$>$

$<$bitrate$>$6000000$<$/bitrate$>$

$<$quant$>$3$<$/quant$>$

$<$aquant$>$5$<$/aquant$>$

$<$txtquant$>$2$<$/txtquant$>$

$<$/mpeg2$>$

$<$h264$>$

$<$bitrate$>$10000000$<$/bitrate$>$

$<$quant$>$10$<$/quant$>$

$<$aquant$>$14$<$/aquant$>$

$<$/h264$>$


\newpage


\section{Appendix B\: Customers by SGID}
\label{APPENDIX-CustBySGID}




\begin{table}
\begin{tabular}{|l|l|l|} 
\hline Market & SGID & Customer Count \\
\hline  &  &  \\  
\hline  &  &  \\
\hline  &  &  \\
\hline  &  &  \\
\hline  &  &  \\
\hline  &  &  \\
\hline  &  &  \\
\hline  &  &  \\
\hline  &  &  \\
\hline  &  &  \\
\hline  &  &  \\
\hline  &  &  \\
\hline 
\end{tabular}
\caption{\label{TABLE-HighSGUse} Service groups with customer counts over 720.} 
\end{table}



CustPerSG/edprmnResults.txt

    744 40019 

    819 30003 

    850 41010 

   1006 40017 

   1079 30006 

   1083 30001 



CustPerSG/knwdmiResults.txt

    726 49011 

    736 37025 

    748 21032 

    766 48076 

    799 28022 

    805 13010 

   1222 48075 



CustPerSG/mddcwiResults.txt

    731 14034 

    740 42028 

    741 24011 

    771 52031 

    786 29002 

    791 36021 

    817 27017 

    833 23004 

    877 42035 

   1014 13029 

   1212 50002 



CustPerSG/pldcorResults1.txt

    729 40225 

    739 40956 

    755 14022 

    763 40186 

    766 20017 

    779 20019 

    789 40114 

    794 40518 

    802 15001 

    806 27068 

    815 27070 

    828 14018 

    834 27022 

    836 40116 

    842 20012 

    843 20003 

    908 14023 

    991 14026 

    992 40154 

   1048 14030 

   1049 14016 

   1087 14035 

   1413 14020 

   1415 14019 

   1878 14007 

 

CustPerSG/pldcorResults2.txt

    723 28011 

    727 38023 

    733 25005 

    733 26003 

    737 25002 

    740 38034 

    746 25006 

    747 28009 

    749 28008 

    767 28006 

    768 19002 

    768 43017 

    774 38035 

    782 43016 

    791 43013 

    793 25019 

    798 25020 

    811 34004 

    822 25008 

    827 19004 

    837 43035 

    843 43011 

    849 25011 

    858 25021 

    864 19012 

    880 26002 

    882 25010 

    901 34002 

    902 43012 

    921 25012 

    940 38014 

    959 19008 

   1006 38030 

   1014 25004 

   1073 19007 

   1077 34005 

   1159 28002 

   1200 43015 

   1382 34003 



CustPerSG/renonvResults.txt

    721 11018 

    728 11101 

    730 12029 

    730 14015 

    741 12005 

    741 12010 

    749 10031 

    749 11150 

    752 17001 

    757 16010 

    760 10005 

    760 11126 

    766 14016 

    769 11134 

    770 16015 

    775 17002 

    782 11142 

    789 11175 

    797 11149 

    797 11156 

    797 12013 

    804 11147 

    804 15011 

    813 11044 

    815 12018 

    817 11141 

    823 11068 

    830 11094 

    832 10050 

    840 11151 

    842 11045 

    848 11165 

    857 15022 

    866 11114 

    869 11174 

    876 11179 

    881 12020 

    897 16001 

    904 11140 

    931 11123 

    952 12024 

    956 11059 

    973 12025 

   1008 17010 

   1013 11144 

   1017 12009 

   1023 16016 

   1038 11028 

   1122 11104 

   1208 12008 



CustPerSG/sldclaResults.txt
   - -


CustPerSG/sldcmoResults.txt

   3628 24090 



CustPerSG/spdcscResults.txt

    720 61105 

    734 26077 

    737 22033 

    737 26024 

    752 24061 

    757 26089 

    758 32093 

    782 30015 

    794 22028 









\end{document}

