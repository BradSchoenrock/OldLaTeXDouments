\documentclass{article}
\usepackage{amsmath,amssymb}
\usepackage[margin=1.0in]{geometry}
\title{A Guide to: Relics of the Big Bang}
\author{Brad Schoenrock and Relics staff}
\date{}
\begin{document}
\maketitle
\section{greetings}
Hello planetarian, and thank you for considering showing Relics of the Big Bang at your planetarium. Here you will find all the pieces we have to offer to program this show with your system. Ultimately your system will not be exactly the same as ours at Abrams, so some judgment must be used in deciding what visuals to use straight out of the box, which ones might need some slight modification to look their best with your system, and which ones will need to be scraped and redone entirely. Our main goal is to make the show look and sound as good as possible while maintaining the scientific and creative integrity of the show necessary to entertain and inform the public. As you proceed i hope you will have this same goal. 
\section{Abrams setup}
Of course you know that every planetarium setup is unique, this section is devoted to informing you of the system that was originally used to program the show. 
***********
\begin{enumerate}
\item L-H-S level specification 5
\item digistar 2
\item 1 video projector
\item SMPTE time code based automation system
\item mackie sdr24 audio playback device
\item spice automation 3.26
\item 6 projector allsky system
\item 6 panel panorama 180 degrees
\item dissolving pairs of "left right center" slide projectors
\item 1 wide angle slide projector
\end{enumerate}
L-H-S level specification 4 or better is recomended to run this show. 
\section{Contents of this Package...}
\begin{enumerate}
\item Disk 1: video-DVD
\item Disk 2: audio-CD
\item Disk 3: digistar commentary-DVD
\item Disk 4: CD data disk containing: slides, panoramas, allskies, digistar files, an example spice file, the master script in .doc and .pdf format, a list of visuals in .xlsx and .pdf format, and other neat info like this! 
\end{enumerate}
\section{tips}
\subsection{Disk 1: the video}
For this show there are several interviews. Some have their audio on the audio disk, while others have their audio on the DVD. The ones that have their audio on the DVD are the ones that have video of the interviewee. Megan Donahue, Mark Voit, Kirsten Tollefson, James Koll, and Patrick True were interviewed at Abrams planetarium and consequently we have their pretty mugs on camera. Barbara Alverez and Wojtek Fedorko are post docs stationed in Geneva and consequently we have no video for these interviewees. The reason for having the audio on the DVD is to ensure that the video and the audio sync up for their interviews without overdue effort. While these interviews are happening there is still background music on the main audio track. Look in the file VisualsForRelicsatAbrams.xlsx for approximate times that each clip starts and ends. Some of the clips are quoted together because they play back to back. We did make our best attempts to make the video pop-in-and-play-ready, but as we developed the show for distribution some of the times didn't work out exactly right. This means that you may have to search the video disk from time to time to ensure that the videos come in when they are intended. 
\subsection{Disk 2: the audio}
The audio is a simple stero mix. If a 5.1 suround sound mix is something that you are interested in then contact abrams planetarium.  
I think this is pretty straightforward, if you listen to the audio off the disk you will hear 8 beeps to begin things off, the first 4 are to try to sync up the right and left inputs, and the second 4 are to listen if you got it right with the final beep siginifying the 0:00:00 time stamp... 
All audio was done by Luke Schwarzweller, who did an exceptional job in spite of some truly adverse working conditions for his position. 
\subsection{Disk 3: Digistar commentary} 
We have included a video with commentary on the digistar, so if you don't have a digistar system you at least know what we did on our system allowing you to replicate it as best you can. If you have a digistar the files have been included on disk 4. Also note that this was created with a few scenes that have strong references to MSU that were eventually cut from the show, so if you are watching it and find something not referenced in the script DONT PANIC. 
\subsection{Disk 4: slides, pans, allskies, and other fun stuff}
We have added all the allskies and panoramas that we used in the show. One interesting thing is allsky for box 110 is given as a 360 degree panorama as well as cut up allsky. We lined it up and turned it into an allsky because we don't have a 360 pan. If you have a 360 pan setup this would probably be a better choice for showing this. All other allskies are given in full dome view and as six wedges. There is one mask cut up in the allsky folder as well; you will of course need a set of 6 for each allsky you use. (You can of course cut them up yourself if your allsky projectors aren't lined up symmetrically) 
\newline
\newline
We have also supplied some extra slides with suggestions for where they could be included implied in their names. 
\newline
\newline
Digistar: files have been included here. The main file is mainrelics.sfa. there are other supporting files in the tar file, as well as some extraneous things which we didn't feel like sorting out... :) best of luck! (If you have any questions contact John French frenchj@msu.edu) 
\newline
To install the digistar files, move the file relics.tar.gz to whatever folder you will be using, then open a terminal and type the following command...
\newline
\newline
tar -zxvf relics.tar.gz
\newline
\newline
…this will uncompress the tar file. 
\newline
\newline
Spice file: we have included our spice file to help you with your automation system. note this includes some sections that were eventually cut from the show, but it will give you perhaps the first 5-10 min of the show and more importianly an idea of how to proceed. 
\newline
\newline
Also included is a questionnaire about the show, if you could have people voluntarily fill these out and return the results to Abrams planetarium, we would greatly appreciate it. 
\section{contact info}
Contact information for Abrams planetarium:
\newline
\newline
Address:
\newline
Abrams Planetarium
\newline
Michigan State University
\newline
East Lansing, MI 48824
\newline
\newline
Phone --- 517-355-4676 during regular business hours.
\newline
\newline
Website --- http://www.pa.msu.edu/abrams/
\newline
\newline
Contact email --- frenchj@msu.edu
\section{conclusion}
Once again I would like to personally thank you for considering Relics of the Big Bang and remember, if the audience enjoys watching this show half as much as we have enjoyed making it, then we enjoyed it twice as much as them. 
\newline
\newline
Thank You
\newline
\newline
Brad Schoenrock, Director. 
\end{document}


