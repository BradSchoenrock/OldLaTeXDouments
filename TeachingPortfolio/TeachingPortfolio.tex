\documentclass{article}
\usepackage{amsmath,amssymb}
\usepackage[margin=1in]{geometry}
\title{Teaching Portfolio}
\author{Brad Schoenrock}
\date{}
\begin{document}
\maketitle
\Large

\tableofcontents

\newpage

\section{Curriculum Vitae}

\subsection{Education}
\hspace{0.5cm}PhD seeking Graduate Student MSU 2010-present

Masters Michigan State University 2012

REU at Cornell University 2009

B.S. physics and mathematics with minors in chemistry and computer science Northern Michigan University 2006-2010


\subsection{Research experience}
\hspace{0.5cm}Search for tZ final state in ATLAS, Advisor: Dr. Reinhard Schwienhorst

Calculation of Diboson cross sections at the LHC, Advisor: Dr. Reinhard Schwienhorst

Single top quark cross section measurement in the t-channel at the high-luminosity LHC, Snowmass 2013,  Advisor: Dr. Reinhard Schwienhorst

Creating the planetarium show "Relics of the Big Bang" for ATLAS outreach, Advisor: Dr. Reinhard Schwienhorst

Fitting reaction rates for astrophysically relevant nuclear processes, Advisor: Dr. Hendrik Schatz. 

Assembly of neutron detectors for a measurement of the electric form factor of the neutron, Advisor: Dr. William Tireman

Soil and sediment sampling project (Searching for Naturally Occurring Radioactive Materials) in Marquette MI, USA, Advisor: Dr. William Tireman


\subsection{Publications and Papers}
\hspace{0.5cm}Single top quark cross section measurement in the t-channel at the high-luminosity LHC
Brad Schoenrock, Elizabeth Drueke (Michigan State U.), Barbara Alvarez Gonzalez (Michigan State U. \& Cantabria Inst. of Phys.), Reinhard Schwienhorst (Michigan State U.). Aug 28, 2013. 16 pp.
e-Print: arXiv:1308.6307

Snowmass 2013 Top quark working group report
Top Quark Working Group Collaboration (K. Agashe (Convener) et al.). Nov 8, 2013. 54 pp.
e-Print: arXiv:1311.2028 

Searches for resonances in the tb and tc final states at the high-luminosity LHC
with contributions from the ATLAS Collaboration (Elizabeth Drueke et al.). Sep 26, 2013.
e-Print: arXiv:1309.7043


\subsection{Presentations}
\hspace{0.5cm}"Single top quark cross section measurement in the t-channel at the high-luminosity LHC" Snowmass proceedings, 2013

"Relics of the Big Bang-Live talks" Abrams planetarium, 2011

"Developing Automatic Test Electronics for Measuring the Performance of SRF Cavities" at Argonne National Laboratory’s Symposium for Undergraduates, 2009

"Neutron Detector Assembly for a High Precision Measurement of the Electric Form Factor of the Neutron" at Argonne National Laboratory’s Symposium for Undergraduates, 2008


\subsection{Languages}
\hspace{0.5cm}I am familiar with C++, root, LaTex, java, labview, MATLAB, some bash, some Visual Basic, and some assembly code. 


\subsection{Awards and Acknowledgements}
\hspace{0.5cm}AAPT Best Graduate TA, 2012-2013

Graduated Suma Cum Laude NMU, 2010

Honors Graduate, Northern Michigan University, 2010

Outstanding physics undergraduate Northern Michigan University, 2010


\subsection{Job experience}
\hspace{0.5cm}Research assistant MSU high energy physics group, 2010-present

Teaching Assistant, MSU physics department, Introductory Physics Laboratory 1, 2010

Research assistant National Superconducting Cyclotron Laboratory, 2010

Walk in tutor at NMU, Marquette MI, 2008-2010

Research assistant physics dept. at NMU, Marquette MI, 2008

Individual tutoring for Student Support Services (SSS) at NMU, 2007-2010

Walk in tutor for SSS, Marquette MI, 2007-2010

Study group tutor for SSS covering introductory physics 1 \& 2, Marquette MI, 2007-2010

Lab setups for the physics dept. at NMU, Marquette MI, 2007-2010


\newpage
\section{Teaching Philosophy}
My goal in teaching would be to implement methods that allow for all learning types to have opportunities to succeed. I would do this by ensuring plenty of demonstrations of physical principles at work and getting the students involved in these demonstrations. Part of what makes physics unique is that what we teach in the classroom will reflect reality and can be readily tested with the simplest of setups. This means there are many ways to incorporate demonstrations into lectures. It is my experience that these techniques work best in small classrooms(such as the ones I learned in at Elcho High School and Northern Michigan University), where the student has a chance to be engaged and get to know the professor, rather than being able to hide in the corner in a class of a hundred students. Through these active learning techniques, small classrooms, and proper curriculum development students should be able to learn science the way science is really done, by direct observation of these demonstrations in class, questioning what they saw, coming up with ideas for how to explain what they saw, testing their predictions, and coming to conclusions about the validity of the hypothesis they themselves came up with. In utilizing this methodology students will not only understand what is covered in class better, but will be better prepared for more advanced class work and future employment. 

Larger classrooms are unfortunatly a nessecity in modern academia, and need to be handled delicatly. many of the techniques that I have found do not "scale up" very well into a large classroom, often coming across as a gimmick to the students who know that it isn't the focus of that lecture and are often ignored by the students. I have found that the best way to combat this is to give small 0-1-2 style graded assignments every day in every class. This works in two ways. First and formost it ensures that students are coming to class and it offers a tool for formative assessment of the students in a variaty of ways. If the small assignment changes format frequently it keeps the students engaged and interested to see what they will be doing today, and in so doing, forcing your students to work outside their comfort zone becomes crucial for this methodology. 


\newpage
\section{Teaching Experiences}
My teaching experience started young with tutoring math and science to my classmates in high school and those younger than me in middle school. My tutoring experience continued throughout college where I tutored students with disabilities with Student Support Services, ran study groups, and did walk in tutoring for All Campus Tutoring. In graduate school I was a TA for introductory physics libratory where I was responsible for administering quizzes and grading lab reports. Continuing with teaching I have begun the certification of college teaching program at MSU.

\section{Professional Development}
As a tutor at Northern Michigan University for Student Support Services I underwent formal tutor training and weekly meetings where we reviewed tutoring methodology. The SSS Tutor Training program is certified by the College Reading and Learning Center Association (CRLA). 

I served on the committee for graduate student curriculum development in the department of Physics and Astronomy where we discussed the feasability of merging graduate courses to expidite graduate student coursework. 

I have also begun the certification of college teaching program at MSU which includes a formal class on teaching college science, a two day seminar on teaching college science, and a mentored teaching experience. As part of this certificate i created a series of intoductory viedos to our intro physics lab for nonmajors, evaluating the educational objectives and tailoring the videos to achieve those objectoves. Also in the course of compleating this certificate i wrote a sylabus and a lesson plan for a modern physics coruse (attached). 


\newpage
\section{Teaching Evaluations}
\hspace{0.5cm}The most adequate teaching evaluations that I can report on are the student evaluations from my semester of TA work. Fall of 2010 I was the TA for PHY 251 which is introductory lab for nonmajors at MSU. 

The Evaluations were broken up into two sections, a short answer and free response. The free response section had the following questions. 

\begin{enumerate}
\item What helped you learn?
\item What hindered (got in the way) of your learning? 
\item What questions do you have that you would like answered?
\item What changes would you suggest to improve your learning?
\item Would you recommend this course to other students? Explain.
\item Other comments.
\end{enumerate}

\subsection{compatibility with Code of Teaching Responsibility}
\hspace{0.5cm}One of the questions was "The labratory was conducted in a manner consistent with the Code of Teaching Responsibility". This was ranked on a scale of 1-5 (1=strongly agree, 5=strongly disagree). The results were a mean of $\mu=1.68$ with a standard deviation of $\sigma=0.66$. so a perfect score is just barely out of the one standard deviation range. One reason for this is that many of the students didn't like the fact that the course was graded on a curve. 

\subsection{Grade the TA}
\hspace{0.5cm}Another (perhaps more relevant) question was on a 4.0 scale grade your TA. the results were a mean of $\mu=3.44$ and a standard deviation of $\sigma=0.69$. As we teach our students, that measurement is compatible with a 4.0. 

\newpage
I have also compiled the following responses that contained the words TA, Brad, teacher, explain, questions, and examples. I chose those words to ensure the responses here were related to my work as a TA. 

\subsection{What helped you learn?}
\hspace{0.5cm}The TA

The TA going over concepts before lab

Having the TA work through things with me

TA explaining the content

The TA

Explanation of questions

Being able to ask questions

Brad, our TA

The TA going through the examples

Asking for help

Explanations in the beginning of lab helped clarify the manual

Brad the TA was very helpful when we had questions and he explained everything really well

Watching the TA do examples with everyone watching

My TA was very helpful and did a great job

The lab manual and the TA

The TA was very helpful when he explained things and answered questions. 

\subsection{What changes would you suggest to improve your learning?}

\hspace{0.5cm}I don't know, letting the TA's help us more

Make the teacher more available

\subsection{Other comments}

\hspace{0.5cm}Brad did a great job helping us understand the experiments

The TA, Brad, was great. He was willing to answer my questions and clarified when i didn't understand something.

Good job :)

\end{document}
