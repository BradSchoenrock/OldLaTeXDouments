\documentclass{article}

\usepackage[margin=1in]{geometry}
%\setlength{\headheight}{0in}
%\setlength{\headsep}{0in}
\title{Certificate in College Teaching Mentored Teaching Project Proposal:\\Creating Introductory Videos for PHY 251 Lab}
\author{Brad Schoenrock}
\date{}
\begin{document}
\maketitle
\Large


\section{Course Description}

This project will focus on PHY 251 introductory lab for nonmajors. These students are typically freshmen/sophomore students who are not in physics but are in the sciences and/or premed students. With this there is some expectation for the demonstration of critical thinking skills resembling the scientific method. PHY 251 is a once a week class for 1 hour and 50 minutes in which students take a quiz, listen to a short introduction to the lab that week, collect data, analyze the data, and fill out a worksheet on the material that was covered that week. Over the course of the course students are expected to become familiar with data taking strategies, uncertainties, creating histograms and plots, discussing significance of results, and to observe the general physics principals that are taught in the lecture class for nonmajors (PHY 231). The instructor for PHY 251 in the fall 2012 semester will be Jim Linnemann, and in the spring 2013 will be Joey Huston. I will also be working with Gerd Kortemeyer and Richard Halstein for advice and guidance. My graduate advisor is Reinhard Schwienhorst.


\section{Problem}

For this project I would like to address this question of preparedness. Many students arrive in lab with skewed understanding of that week’s lab procedure and what they will be expected to do. This leads to students running out of time in the 1 hour 50 minute lab and this can be frustrating for the students as well as the TA's and instructors. To a certain extent this problem is because they don't read the manual before they arrive and are therefore underprepared. The argument could be made that if they didn't want to read the manual, then they will not participate in other excersizes before lab, but i would argue these videos are not focused on those students. The target audiance is the student who is trying hard but isn't getting the help from the lab manual or their TA's and instructors. i hope that creating introductory videos will help these students who, through no fault of their own, are having a hard time with the course. 


\section{Research Question}

There are three complementary goals to this project. 

\begin{enumerate}
\item Will student preparedness increase if they are given videos to help them prepare? 
\item Will student understanding improve if they are given videos to help them prepare? 
\item Will student opinions on the course change if they are given videos to help them prepare? 
\end{enumerate}


\section{Hypothesis}

I believe that one factor for why the students are underprepared for lab is because the lab manual either isn't giving a clear understanding of what the equipment is, the manual isn't clear on how to use it, or perhaps both. This is what the TA is supposed to be for in each section (to clear up a misunderstandings introduced by the manual) but there is always the problem of the stereotypical bad TA who isn't willing or able to provide the necessary introduction to lab. In this case my hope is that these videos will serve as an augmented introduction to the lab to allow students to become familiar with the equipment and procedures independently of their TA's introduction and the lab manual. I also think that these videos have the potential to be entertaining, and therefore more engaging, which might lead to higher participation. 


\section{Procedure}

During the fall semester I will record myself doing an intro to each lab, with a demonstration of the lab equipment properly set up, pointing out some common mistakes that are often made. These videos will be hosted online for viewing before the spring semester, when they will be implemented and ready to use and assess. 
My plan is to have students, on a voluntary basis, watch the videos before they get into lab. I don't think that watching these videos should in any way be linked into student grades (except in that watching the videos might help prepare them better). That way we will not change the format of the lab section allowing previous semesters to be used as a control group. 

The following data will be collected...

\begin{enumerate}
\item we can poll TA's to get their impression of student understanding. 
\item we can get the students grades as well as poll them if they watched the videos and if they thought they helped. 
\end{enumerate}

Success will be determined by...

\begin{enumerate}
\item positive feedback from TA's.
\item positive feedback from studnents.
\item increased grades among those who watched the videos.
\end{enumerate}


\end{document}














