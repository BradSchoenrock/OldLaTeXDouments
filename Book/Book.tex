\documentclass[oneside]{book}
\usepackage{amsmath,amssymb,url}
\author{Brad Schoenrock}
\title{The Story of the Mountain. (working title)}
\begin{document}
\maketitle

\newpage

\tableofcontents

\newpage

\chapter{Intro thing.}

"Dol? What are you asking about him for? He always was a dick." Itzea may have been charming to many of the menfolk in town, but she certainly lacked tact when it came to those she knew well. 

"Oi, fuck off!" exclaims a dwarf from a corner booth as he throws a mostly empty tankard across the room over Itzea seated at the larger central table of the tavern and over the bar. That alchohol fueled lack of charm and tact must be him. 

The bartender deftly catches the tankard at an awkward angle before it hits the mirror behind him, but maintains his composure and says "Careful Dol, we don't want a repeat of last week." as he nods to a broken table leaning against a wall on the far side of the room.

From the corner Dol grunts, takes the shot sitting in front of him, and sinks on the splintered bench mounted on the wall. 

"Thanks for making my point Dol...." Itzea remarks, "What was it you wanted to know? Right, the story of what happened to the mountain, not much to tell i'm afraid." That was bullshit to be sure, half the mountain was missing plain as day, so something must have happened. "That didn't have much to do with Dol anyway, so why ..." Itzea is interupted by a now empty shot glass as she deflects it from hitting her head sending it sailing into the bar and causing it to break into several pieces. 

With this the bartender comes out from behind the bar. Dol protests without putting up much of a real fight, "But you heard what she said!"

"Come on, you know the rules. We've been over this, you cant go throwing things, whose going to clean up after you...." the bartender talks him down as he escorts Dol out the back door and out of earshot. 

Itzea giggles to herself, "That always gets the egotistical bastard going. So you want to hear the story of the mountain. I guess i can tell it again, lord knows i've told it enough times." She sighs to herself as she crosses her legs and looks discouraged at her nearly empty glass. After a pause "I might need another drink though" she barks back to the bartender as he returns from the back without Dol. Itzea looks back at Dol's absence for a moment, slightly disconcerted, then turns back as the bartender readies an old fashioned for her. "mmm, nothing like an old fasioned amiright." 

The bartender replies as he brings Itzea her drink, "Whatever you say. Maybe just leave Dol alone for a bit and stop provoking him, hm? Ya got him all worked up and now he won't want to deal with anyone for a spell." He walks to the bar to clean up the broken glass, and takes a moment to assess the new scratch in the bar. 

"You know i can't resist giving that old coot a hard time, now where was i ..." 



\newpage

\chapter{Ordinary World}

This is where the Hero's exists before his present story begins, oblivious of the adventures to come. It's his safe place. His everyday life where we learn crucial details about our Hero, his true nature, capabilities and outlook on life. This anchors the Hero as a human, just like you and me, and makes it easier for us to identify with him and hence later, empathize with his plight.

Insert cliche storytelling device here. 

Introduce the other party members here. 



\newpage

\chapter {Call To Adventure}

The Hero's adventure begins when he receives a call to action, such as a direct threat to his safety, his family, his way of life or to the peace of the community in which he lives. It may not be as dramatic as a gunshot, but simply a phone call or conversation but whatever the call is, and however it manifests itself, it ultimately disrupts the comfort of the Hero's Ordinary World and presents a challenge or quest that must be undertaken.



\newpage

\chapter{Refusal Of The Call}

Although the Hero may be eager to accept the quest, at this stage he will have fears that need overcoming. Second thoughts or even deep personal doubts as to whether or not he is up to the challenge. When this happens, the Hero will refuse the call and as a result may suffer somehow. The problem he faces may seem to much to handle and the comfort of home far more attractive than the perilous road ahead. This would also be our own response and once again helps us bond further with the reluctant Hero.



\newpage

\chapter{Meeting The Mentor}

At this crucial turning point where the Hero desperately needs guidance he meets a mentor figure who gives him something he needs. He could be given an object of great importance, insight into the dilemma he faces, wise advice, practical training or even self-confidence. Whatever the mentor provides the Hero with it serves to dispel his doubts and fears and give him the strength and courage to begin his quest.



\newpage

\chapter{Crossing The Threshold}

The Hero is now ready to act upon his call to adventure and truly begin his quest, whether it be physical, spiritual or emotional. He may go willingly or he may be pushed, but either way he finally crosses the threshold between the world he is familiar with and that which he is not. It may be leaving home for the first time in his life or just doing something he has always been scared to do. However the threshold presents itself, this action signifies the Hero's commitment to his journey an whatever it may have in store for him.



\newpage

\chapter{Tests, Allies, Enemies}

Now finally out of his comfort zone the Hero is confronted with an ever more difficult series of challenges that test him in a variety of ways. Obstacles are thrown across his path, whether they be physical hurdles or people bent on thwarting his progress, the Hero must overcome each challenge he is presented with on the journey towards his ultimate goal.

The Hero needs to find out who can be trusted and who can't. He may earn allies and meet enemies who will, each in their own way, help prepare him for the greater ordeals yet to come. This is the stage where his skills and/or powers are tested and every obstacle that he faces helps us gain a deeper insight into his character and ultimately identify with him even more.



\newpage

\chapter{Approach To The Inmost Cave}

The inmost cave may represent many things in the Hero's story such as an actual location in which lies a terrible danger or an inner conflict which up until now the Hero has not had to face. As the Hero approaches the cave he must make final preparations before taking that final leap into the great unknown.

At the threshold to the inmost cave the Hero may once again face some of the doubts and fears that first surfaced upon his call to adventure. He may need some time to reflect upon his journey and the treacherous road ahead in order to find the courage to continue. This brief respite helps the audience understand the magnitude of the ordeal that awaits the Hero and escalates the tension in anticipation of his ultimate test.



\newpage

\chapter{Ordeal}

The Supreme Ordeal may be a dangerous physical test or a deep inner crisis that the Hero must face in order to survive or for the world in which the Hero lives to continue to exist. Whether it be facing his greatest fear or most deadly foe, the Hero must draw upon all of his skills and his experiences gathered upon the path to the inmost cave in order to overcome his most difficulty challenge.

Only through some form of "death" can the Hero be reborn, experiencing a metaphorical resurrection that somehow grants him greater power or insight necessary in order to fulfill his destiny or reach his journey's end. This is the high-point of the Hero's story and where everything he holds dear is put on the line. If he fails, he will either die or life as he knows it will never be the same again.



\newpage

\chapter{Reward (Seizing The Sword)}

After defeating the enemy, surviving death and finally overcoming his greatest personal challenge, the Hero is ultimately transformed into a new state, emerging from battle as a stronger person and often with a prize.

The Reward may come in many forms: an object of great importance or power, a secret, greater knowledge or insight, or even reconciliation with a loved one or ally. Whatever the treasure, which may well facilitate his return to the Ordinary World, the Hero must quickly put celebrations aside and prepare for the last leg of his journey.



\newpage

\chapter{The Road Back}

This stage in the Hero's journey represents a reverse echo of the Call to Adventure in which the Hero had to cross the first threshold. Now he must return home with his reward but this time the anticipation of danger is replaced with that of acclaim and perhaps vindication, absolution or even exoneration.

But the Hero's journey is not yet over and he may still need one last push back into the Ordinary World. The moment before the Hero finally commits to the last stage of his journey may be a moment in which he must choose between his own personal objective and that of a Higher Cause.



\newpage

\chapter{Resurrection}

This is the climax in which the Hero must have his final and most dangerous encounter with death. The final battle also represents something far greater than the Hero's own existence with its outcome having far-reaching consequences to his Ordinary World and the lives of those he left behind.

If he fails, others will suffer and this not only places more weight upon his shoulders but in a movie, grips the audience so that they too feel part of the conflict and share the Hero's hopes, fears and trepidation. Ultimately the Hero will succeed, destroy his enemy and emerge from battle cleansed and reborn.



\newpage

\chapter{Return With The Elixir}

This is the final stage of the Hero's journey in which he returns home to his Ordinary World a changed man. He will have grown as a person, learned many things, faced many terrible dangers and even death but now looks forward to the start of a new life. His return may bring fresh hope to those he left behind, a direct solution to their problems or perhaps a new perspective for everyone to consider.

The final reward that he obtains may be literal or metaphoric. It could be a cause for celebration, self-realization or an end to strife, but whatever it is it represents three things: change, success and proof of his journey. The return home also signals the need for resolution for the story's other key players. The Hero's doubters will be ostracized, his enemies punished and his allies rewarded. Ultimately the Hero will return to where he started but things will clearly never be the same again.



\newpage

\appendix

\chapter{notes}

Act 1: Itzea narrates in first person

Act 2: Dol narrates the climax in first person

Act 3: they interject along with Bob (third person commentary) to conclude the story. 

Mood: Feelings of the reader
Tone: Feelings of the narrator



\chapter{Characters}

Reader: silent nondescript observer being told the story by the characters

Dolkuhm (Dol): cantankourus as fuck drunkard, in your face, almost yelling the story at you, kind of obnoxious sometimes and you want him to go away. 

Itzea: no nonsense whore who chooses her time and her clients, but can tell a good story, practical yet flowing language, knows how to get under people's skin when she wants to but chooses not to most of the time

Bartender Bob: Just wants to be left alone and run his bar, stop giving him shit, knows the party well but didn't partake in the adventure, mentor type former adventurer 

Party member 1: charmer, sitting in a corner with a woman the whole time, doesn't contribute his perspective to the narritive in the pub but will be active in the story. 

Party member 2: noble type paladin, uses knights cant, not present in the pub but will enter late in the story as a call to action for their next adventure. 



\chapter{Bob's chapter}

Following up on that thought, choose one character to reveal a certain truth. Choose a character who has a different experience that’s not common to the others. Always ask yourself:

Why am I using this particular character to tell this part of the story?
What insight or awareness does the character bring?
Is this experience best told through this character’s point of view?



\end{document}


