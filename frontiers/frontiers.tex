\documentclass{article}
\usepackage{amsmath,amssymb,url}
\usepackage[margin=1in]{geometry}
\author{Brad Schoenrock}
\title{Frontiers: Responsible Conduct in Research}
\date{}
\begin{document}
\maketitle
\Large

The book "Voodoo Science" covers many ways people can use science to deceive or to be deceived. Here I sum up some of the main points the book makes. 

\section{Being Fooled}

The media has always been terrible at covering scientific news. This has been a great source of frustration for scientists who only want to understand the truth. The problem with news media is that they don't seek explanations that the general population can use to become more informed or more interested in science. Instead every topic is reduced to five second sound bites. Furthermore news agencies are always trying to present a picture of unrest or conflict in any issue. giving both sides of an argument can lead to a false sense of opinions being split in the scientific community on important issues like cold fusion, climate change, or perpetual motion machines. It is nice that YouTube has relieved some of the constraints that traditional news media had. Now YouTube channels such as Vsauce, SciShow, and CGPgrey have the time and the means to accurately represent the facts about cool things all around us. 

It is also true that some try to obfuscate the facts through pseudoscience. They make claims that are too good to believe and arguments that are hard or impossible to refute. Vitamin O and homeopathic cures claim to heal many things and have tricked many into purchasing many "medicines" that do nothing. Sometimes an argument can have some compelling arguments while not actually representing the truth. It is easy to be fooled if you aren't careful, and even the most vigilant scientists find themselves fooled from time to time by some of these tricks. 

\section{Fooling Yourself}

"In science, it is not good to want something too badly" V. Zelevinsky 

I think this quote from my quantum mechanics instructor sums up much of the book quite well. It is easy to believe something is true because we were told it when we were young, or because we saw it that it work that one time. We all make causal relationships like that every day. When you were a child and touched a hot stove you didn't try it again to prove that you should not put your hand on a hot stove. You accept it and move on. Sometimes that works, while other times it can mislead you into believing something like wearing your "lucky hat" will help your sports team win. Most people cling to these types of beliefs not because it works every time, but because it worked once and then every time it didn't work they explained it away with an excuse. They aren't dissuaded by all the evidence to the contrary, because in their minds all the evidence to the contrary isn't entirely dissuasive. This leads to two important rules scientists must be willing to follow to retain objectivity. 

\begin{enumerate}
\item Expose new ideas and results to independent testing and replication by other scientists.
\item Abandon or modify accepted facts or theories in the light of more complete or reliable experimental evidence. 
\end{enumerate}

Following these two rules will ensure you don't use science to prove you're right; you will use science to become right. If everybody follows these rules then everybody gains from science, because whether we were right to begin with or not, in the end we all will be. 
This is particularly problematic in situations where science contradicts widely accepted societal beliefs or deeply rooted religious beliefs. 

\section{Fooling Others} 

Finally there is the problem of fooling others. It is often hard to accept a mistake made either in haste or made to explain an unexpected result. This is likely how many scientific hoaxes start. 

Fleischmann and Pons, for example, most likely thought they made a groundbreaking cold fusion discovery at first. After a certain period of peer review when their result had been largely discredited they mad several errors. They stuck to their faulty result, failed to release details of their experiment, and generally tried to cover up their mistake. In trying to protect themselves they found that they had crossed into a world of fraud from which there was no return. 

Another example is Neuman's perpetual motion machine that involved a self taught man thinking he knew more than educated scientists. The takeaway here is twofold, when a mistake is first discovered you should come out with it right away, and be sure to do your homework. If Newman had understood the first law of thermodynamics or back EMF at all, he would have known the design wouldn't work. All he had to do is open a textbook on either of these subjects and it would have spared him the embarrassment of getting shot down by John Glenn in a congressional hearing. 

In summary, be vigilant so as not to get fooled by others, be mindful of your own biases, and be humble enough to admit a mistake. 










\begin{thebibliography}{9}
\bibitem{thingy} Park, Robert L., Voodoo Science, \textbf{1} (2000), Oxford University Press
\end{thebibliography}
\end{document}